% myarticle.tex @ https://github.com/zfengg/toolkit/tree/master/tex/myarticle
%
% 	A simple tex template based on amsmath-userguide for daily tasks.
%
% Copyright: Zhou Feng @ 08/11/2020
% ---------------------------------------------------------------------------- %
%                                   preamble                                   %
% ---------------------------------------------------------------------------- %
\documentclass[12pt]{amsart} 
% leqno; reqno for changing the position of number of equations

% --------------------------------- pacakges --------------------------------- %
\usepackage{amsmath}
\usepackage{amsfonts} 			  	% \mathscr \mathbb \mathfrak 
\usepackage{amssymb} 			 	% special symbols
\usepackage{mathrsfs}
\usepackage{mathtools}
\usepackage{savesym}
\usepackage{color} 				  	% colors
\usepackage{graphicx} 			  	% include figures
\usepackage{caption} 	 		  	% caption dealing
\usepackage{amscd} 		 		  	% basic commutative diagrams; complex ones: kuvio, XY-pic
\usepackage{float} 		 		  	% table figure positioning
%\usepackage{booktabs} 			  	% thick rules

%---list environments---%
\usepackage[shortlabels,inline]{enumitem} % smart enumeration
%\newlist{inlinelist}{enumerate*}{1}
%---diagram drawing---%
\usepackage{tikz}    % picture drawing
\usepackage{tikz-cd} % powerful commutative diagrams

% ---------------------------------- layout ---------------------------------- %
\textheight 8.in
\textwidth 6.in
%\topmargin -.25in
\oddsidemargin .25in
\evensidemargin .25in
\footskip 4.ex
\parskip 1.ex
\renewcommand{\baselinestretch}{1.0858}
%\usepackage[textheight=8.0in,textwidth=6.0in]{geometry}
%\usepackage{fullpage}	% fit more on a page without having to set margins manually 

% ----------------------------- fonts and colors ----------------------------- %
%\usepackage{fontspec}% Then you can use the fonts installed at your device. 
%\setmainfont{Times New Roman}
%\setsansfont{Times New Roman}
%\setmonofont{Times New Roman}
%\setsansfont{[foo.ttf]} % for the fonts at this default path.
%\usepackage{xcolor}
%\definecolor{MSBlue}{rgb}{.204,.353,.541}
%\definecolor{MSLightBlue}{rgb}{.31,.506,.741}

% ----------------------------- header and footer ---------------------------- %
%\usepackage{fancyhdr}
%%headstyle
%\newcommand{\headstyle}{
%	\fancyhead[L]{}
%	\fancyhead[C]{}
%	\fancyhead[R]{}
%}
%% footstyle
%\newcommand{\footstyle}{
%	\fancyfoot[L]{}
%	\fancyfoot[C]{\thepage}
%	\fancyfoot[R]{}
%}
%\pagestyle{fancy}  % Set pagestyle for the document
%\fancyhf{} % clear the orginal headfoot style
%\headstyle % set the new head style
%\footstyle % set the new foot style
%% Define a new pagestyle for main pages
%\fancypagestyle{main}{%
%	\fancyhf{} % clear the original pagestyle
%	\headstyle
%	\footstyle
%}
%% for the first page
%\fancypagestyle{firstpage}{%
%	\fancyhf{} % clear the orginal pagestyle
%	\fancyhead[L]{\small\itshape Curriculum Vitae}
%	\fancyhead[C]{}
%	\fancyhead[R]{}
%	\fancyfoot[L]{}
%	\fancyfoot[C]{\thepage}
%	\fancyfoot[R]{}
%}
%\renewcommand{\headrulewidth}{0pt}
%\renewcommand{\footrulewidth}{0.4pt}
%\renewcommand{\headrule}{\rule{\textwidth}{0.4pt}}

% ------------------------------ reference style ----------------------------- %
\usepackage[numbers,sort]{natbib} 	% automatically sort the citations
\newcommand{\seeeg}[1]{see e.g.\ #1}
\newcommand{\seep}[1]{(see #1)}
\newcommand{\seepeg}[1]{(see e.g.\ #1)}

% -------------------------- hyperlinks & bookmarks -------------------------- %
\usepackage{aliascnt} % for the correct \autoref
\usepackage[bookmarks=true,
		pagebackref,
		colorlinks,
		linkcolor=blue,
		citecolor=blue,
		urlcolor=black]{hyperref}

% -------------------------------- environment ------------------------------- %
\usepackage{amsthm}
\newtheorem{dummy}{***}[section]

\theoremstyle{plain}
\newaliascnt{theorem}{dummy}
\newtheorem{theorem}[theorem]{Theorem}
\aliascntresetthe{theorem}
\providecommand*{\theoremautorefname}{Theorem} 
\newaliascnt{proposition}{dummy}
\newtheorem{proposition}[proposition]{Proposition}
\aliascntresetthe{proposition}
\providecommand*{\propositionautorefname}{Proposition} 
\newaliascnt{corollary}{dummy}
\newtheorem{corollary}[corollary]{Corollary}
\aliascntresetthe{corollary}
\providecommand*{\corollaryautorefname}{Corollary} 
\newaliascnt{lemma}{dummy}
\newtheorem{lemma}[lemma]{Lemma}
\aliascntresetthe{lemma}
\providecommand*{\lemmaautorefname}{Lemma} 
\newaliascnt{conjecture}{dummy}
\newtheorem{conjecture}[conjecture]{Conjecture}
\aliascntresetthe{conjecture}
\providecommand*{\conjectureautorefname}{Conjecture}

\theoremstyle{definition}
\newaliascnt{definition}{dummy}
\newtheorem{definition}[definition]{Definition}
\aliascntresetthe{definition}
\providecommand*{\definitionautorefname}{Definition} 
\newaliascnt{example}{dummy}
\newtheorem{example}[example]{Example}
\aliascntresetthe{example}
\providecommand*{\exampleautorefname}{Example} 

\theoremstyle{remark}
\newaliascnt{remark}{dummy}
\newtheorem{remark}[remark]{Remark}
\aliascntresetthe{remark}
\providecommand*{\remarkautorefname}{Remark} 

% --------------------------------- counters --------------------------------- %
\numberwithin{equation}{section} % \numberwithin also works for other counters

% --------------------------------- pagestyle -------------------------------- %
%\pagestyle{plain}

% ---------------------------------- autoref --------------------------------- %
\renewcommand{\sectionautorefname}{Section}
%\renewcommand{\equationautorefname}{Equation}	

% ------------------------------------ misc ----------------------------------- %
%\setcounter{MaxMatrixCols}{11} % the maximal cols that a matrix can have
\allowdisplaybreaks

% -------------------------------- fast-typing ------------------------------- %
% cali characters
\newcommand{\calF}{\mathcal{F}}
\newcommand{\calB}{\mathcal{B}}
\newcommand{\calP}{\mathcal{P}}
\newcommand{\calT}{\mathcal{T}}
\newcommand{\calD}{\mathcal{D}}
\newcommand{\calQ}{\mathcal{Q}}
\newcommand{\calR}{\mathcal{R}}
\newcommand{\calM}{\mathcal{M}}
\newcommand{\calL}{\mathcal{L}}
\newcommand{\calI}{\mathcal{I}}
% scurf characters
\newcommand{\scrB}{\mathscr{B}}
\newcommand{\scrD}{\mathscr{D}}
\newcommand{\scrP}{\mathscr{P}}
\newcommand{\scrA}{\mathscr{A}}
\newcommand{\scrE}{\mathscr{E}}
\newcommand{\scrC}{\mathscr{C}}
\newcommand{\scrT}{\mathscr{T}}
\newcommand{\scrQ}{\mathscr{Q}}
% blackboad characters
\newcommand{\bbQ}{\mathbb{Q}}
\newcommand{\bbR}{\mathbb{R}}
\newcommand{\bbC}{\mathbb{C}}
\newcommand{\bbZ}{\mathbb{Z}}
\newcommand{\bbN}{\mathbb{N}}
\newcommand{\bbP}{\mathbb{P}}
\newcommand{\bbT}{\mathbb{T}}
% frak characters
\newcommand{\frX}{\mathfrak{X}}
% rome characters
\newcommand{\rmN}{\mathrm{N}}
\newcommand{\rmR}{\mathrm{R}}
% math operators
\DeclareMathOperator*{\Lim}{Lim}
\DeclareMathOperator*{\intersect}{\cap}
\DeclareMathOperator*{\Intersect}{\bigcap}
\DeclareMathOperator*{\union}{\cup}
\DeclareMathOperator*{\Union}{\bigcup}
\DeclareMathOperator{\ex}{ex}
\DeclareMathOperator{\co}{conv}
\DeclareMathOperator{\diam}{diam}
\DeclareMathOperator{\supp}{supp}
% commonly used
\newcommand{\Ndash}{\nobreakdash--}
\newcommand{\sigmaAlg}{$\sigma$\nobreakdash-algebra }
\providecommand{\abs}[1]{\lvert#1\rvert}
\providecommand{\Abs}[1]{\left\lvert#1\right\rvert}
\providecommand{\norm}[2][]{\lVert#2\rVert_{#1}}
\providecommand{\Norm}[2][]{\left\lVert#2\right\rVert_{#1}}
\newcommand{\ndiml}[1][n]{$#1$\nobreakdash-dim}
\newcommand{\padic}{$p$\nobreakdash-adic}
\newcommand{\euclid}[1][d]{\mathbb{R}^{#1}}
\newcommand{\innerprod}[2]{\langle #1, #2 \rangle}
\newcommand{\dimH}{\dim_{H}}
\newcommand{\dimP}{\dim_{P}}
\newcommand{\leb}[1]{\mathcal{L}(#1)}
\newcommand{\lebRes}[1]{\mathcal{L}\vert_{#1}}
\newcommand{\energy}[2][s]{I_{#1}(#2)}
\newcommand{\capacity}[2][s]{g_{#1}(#2)}
%\genfrac{left-delim}{right-delim}{thickness}{mathstyle}{numerator}{denominator}

% special for this article

% --------------------------------- titlepage -------------------------------- %
\title[small title]{My Article}
% remove any unused author tags.
% author one information
\author{Zhou Feng}
\address{	Department of Mathematics\\
	The Chinese University of Hong Kong\\
	Shatin,  Hong Kong}
\curraddr{}
\email{\href{mailto: zfeng@math.cuhk.edu.hk}{zfeng@math.cuhk.edu.hk}}
\thanks{}

%% author two information
%\author{}
%\address{}
%\curraddr{}
%\email{}
%\thanks{}

%----for the footnotes of titlepage-----%
%\subjclass[2010]{Primary 28A80, 42A85} 
%\keywords{Self-similar measure}
%\date{}
%\dedicatory{}
%\thanks{}

% ---------------------------------------------------------------------------- %
%                                   document                                   %
% ---------------------------------------------------------------------------- %
\begin{document}
%\begin{abstract}
%	 
%\end{abstract}
\maketitle

% ------------------------------- introduction ------------------------------- %
%\setcounter{section}{0}
%\section{Introduction}\label{sec:Introduction}
%\subsection{Background}\label{subsec:Background}
%
%\subsection{Results}\label{subsec:Results}
%
%\subsection{Strategy of the proof}\label{subsec:Strategy}
%
%\subsection{Notations}\label{subsec:Notations}
%
%\subsection*{Structure of the paper}\label{subsec:StructurePaper}
%
%\subsection*{Acknowledgement}\label{subsec:Acknowledgement}

\section{Introduction}\label{sec:intro}
A good introduction to fractal geometry is Falconer~\cite{Falconer2003}. There is \verb|smallmatrix| environment (e.g, $\big(\begin{smallmatrix}
			a & b\\ c&d
		\end{smallmatrix}\big)$). It is recommended to use $ \dotsc, \dotsb, \dotsm, \dotsi, \dotso $ instead of $ \ldots \text{and} \cdots$. Then we test the \verb|\nobreakdash|: \padic, page 1\Ndash9, \ndiml, \ndiml[1], \sigmaAlg.
What about a text-mode fractional: $ \tfrac{\log_{k} H}{1212} $.\\[20pt]
Then for the \verb|\xleftarrow|:

\begin{equation}\label{eq:arrow}
	A \xleftarrow{n+\mu-1} B \xrightarrow[T]{n\pm i-1 \text{bla, bla, bla}}	C \intersect_{i\geq 1} A_{i} \Union_{k=1}^{100}\Upsilon_{k}
\end{equation}
\[ \Leftarrow \mspace{-18.0mu} \Rightarrow \] % 18mu=1em

Compare the \verb|\choose| and \verb|\binom| : $ {n \choose k} \binom{n}{k}$. $ \abs{z}, \norm{v}, \norm[\infty]{v} $.

About the user-defined math operators:
\[ \ex(\co(A_{i})) \operatorname*{abc}_{x\to 0} \Lim_{n\to \infty}\]
Then the \verb|mod|: $ \gcd(n,m\bmod n); \quad x \equiv y \pmod b, x \equiv y \mod c, x \equiv y \pod d $.

See the following default math environments:

\begin{multline}\label{eq:newton}
	\vec{F} = m\vec{a} \\
	\vec{F} = G \dfrac{m_{1}m_{2}}{r^{2}}
\end{multline}

\begin{subequations}\label{eq:maxwell}
	\begin{gather}
		\nabla\cdot \vec{E} = \varepsilon_{0}\rho \\
		\nabla \cdot \vec{B} = 0 \\
		\nabla \times \vec{E} = -\dfrac{\partial \vec{B}}{\partial t}\\
		\nabla \times \vec{B} = \mu_{0}\varepsilon_{0} \vec{J} + \dfrac{\partial \vec{E}}{\partial t}
	\end{gather}
\end{subequations}

\begin{equation}\label{eq:einstein}
	\begin{split}
		E& =\gamma mc^{2}
		\\
		\calR_{\mu\nu} - \dfrac{\calR}{2}g_{\mu\nu} + \Lambda g_{\mu\nu} & = \dfrac{8\pi G}{c^{4}} T_{\mu\nu}
	\end{split}
\end{equation}

$\bullet$ \verb|\substack{}| and \verb|\begin{subarray}|
\begin{gather}
	\Lim_{\substack{0\leq i \leq m\\ 0<j<n} } P(i,j)\\
	\sum_{\begin{subarray}{r}
			i\in \Lambda \\ 0<j<n
		\end{subarray}} P(i,j)
\end{gather}

$\heartsuit$ \verb|\sideset{text}{right}{symbol}|

\[
	a_0+\cfrac{1}{a_1+\cfrac{1}{a_2+\cfrac{1}{a_3+\cdots}}}
\]

\begin{equation}
	\sideset{_\ast^\ast}{_\ast^\ast}\prod_{n=1}^{\infty} \left.	\begin{aligned}
		\begin{bmatrix}
			\sideset{_\ast^\ast}{_\ast^\ast}\prod_{n=1}^{\infty} & \sum & \prod_{n=1}^{\infty} \\
			                                                     & \ex  & \lim_{n\to \infty}
		\end{bmatrix}
	\end{aligned} \right \rbrace \lim_{n\to \infty} \text{Quantum Computing}
\end{equation}

The \verb|\mathbf| command is commonly used to obtain bold Latin letters in math, but for most other kinds of math symbols it has no effect.

\verb|\mid| and \verb|\mathbin{}|	:$ P(A \mid B) P(A \mathbin{\vert} B )  P(A | B)  $
\begin{enumerate}[(a)]
	\item \verb|f : X\to Y| vs. \verb|f\colon X\to Y|: $ f : X\to Y $ vs. $ f\colon X\to Y $.

	\item \verb|:=| vs. \verb|\coloneqq| : $ :=  $ vs. $ \coloneqq $.

	\item $ \lbrace z : z \in \bbZ \rbrace $ vs. $ \lbrace z \colon z \in \bbZ \rbrace $.

	\item $ v_{1}, v_{2}, \dotsc,v_{n} $ vs. $ v_{1},\dotsc,v_{n} $.

	\item $ f(n) = O(n) $ vs. $ f(n) \text{ is } O(n)$ or $ f(n)\in O(n) $.

	\item $ A\setminus B $ vs. $ A\backslash B $ vs. $ A-B $.

	\item There is a \verb|\,| spacing between integrand and measure
	      \[ \int_{a}^{b} x^2 \,dx \]
	      
	\item Use \verb|Serre et al.\ proved|: Serre et al.\ proved.
	
	\verb|Serre et al. proved|: Serre et al. proved.
	
\end{enumerate}

Here is some practical suggestions for mathematical writting.
\begin{enumerate}
	\item The structures for conditional sentences: \verb|If| \ldots\verb|, then|\ldots; \verb|When|\ldots, \ldots; \verb|For| \ldots, \ldots. No \verb|Let|\ldots. \verb|Then|\ldots!

	\item Avoid using \verb|as| and \verb|for| to introduce reasons after some conclusion.

	\item \verb|Hence|, \verb|Thus|, and \verb|Therefore,| .

	\item \verb|, so| is informal and should be used when the conclusion is short.

	\item A statement that is \verb|assumed| is an axiom, and throughout to be true. Something \verb|supposed| is a hypothesis and more appropriate to introduce a case or an argument by contradition. For example, \verb|Suppose to the contrary that| and \verb|Toward a contradiction, suppose that|.

	\item No $ v $\verb|'s| or $ a_{i} $\verb|'s|.

	\item No \textit{nested} proof environments.

	\item \verb|We induct on n| vs. \verb|We use induction on n|.

	\item Prefer \verb|pairwise| to \verb|mutually|.

	\item No contractions like \verb|can't|, \verb|won't|, etc.
	
	\item Use \verb|\begingroup\allowdisplaybreaks ... \endgroup| to allow the large chunk of math display environments to be broken into pages.
	
	\item Replace \verb|$$ ... $$| with \verb|\[...\]| in \verb|sed|:
	
	\verb|sed '/\$\$/{:x;N;/.*\$\$ *$/!bx;s/\$\$\(.*\)\$\$ *$/\\[\1\\]/}'|

\end{enumerate}

\section{Commutative diagrams}

Arrows \verb|@>>>| \verb|@<<<| \verb|@VVV| \verb|@AAA|. Double lines: \verb|@=|. Null arrows: \verb|@|
\begin{equation}\label{key}
	\begin{CD}
		S^{{\mathcal{W}}_{\Lambda}\otimes T} @>j>> T\\
		@|  @VV{\operatorname{End} P}V\\
		(S\otimes T)/I  @= (Z\otimes T)/J
	\end{CD}
\end{equation}


\verb|tikzcd| is the ultimate answer to a commutative diagram in \TeX.
\[ \begin{tikzcd}
		A \arrow[rd] \arrow[r, "\phi"] & B \\
		& C
	\end{tikzcd}  \]

\section{reference \& citation}
Choose a \verb*|natbib| compatible \verb|\bibliographystyle|, e.g.\,\verb|abbrvnat|, \verb|plainnat|.
\begin{itemize}[$\cdot$]
	\item \verb*|\cite{}|: \cite{AkiyamaEtAl2020}
	\item \verb*|\citet{}|: \citet{AkiyamaEtAl2020}
	\item \verb*|\citet*{}|: \citet*{AkiyamaEtAl2020}
	\item \verb*|\citep{}|: \citep{AkiyamaEtAl2020}
	\item \verb*|\citep*{}|: \citep*{AkiyamaEtAl2020}
	\item \verb*|\citealt*{}|: \citealt*{AkiyamaEtAl2020}
	\item \verb*|\citeyear{}|: \citeyear{AkiyamaEtAl2020}
	\item \verb*|\citeauthor{}|: \citeauthor{AkiyamaEtAl2020}
	\item \verb*|\citeauthor*{}|: \citeauthor*{AkiyamaEtAl2020}
	\item \verb*|\cite[text]{keylist}| \cite[Theorem 1]{AkiyamaEtAl2020}
	\item \verb*|\cite[prefix][suffix]{keylist}|: \cite[see e.g.\!][p.\,123]{AkiyamaEtAl2020}
	\item \verb*|\citenum{}|: \citenum{AkiyamaEtAl2020}
	\item \verb*|\citeyearpar{}|: \citeyearpar{AkiyamaEtAl2020}
	\item \verb*|\citefullauthor{}|: \citefullauthor{AkiyamaEtAl2020}
\end{itemize}

See also a book \citet*{Parry1981} and an arXiv preprint \cite{Feng2020}.
More multi-authors citation like \citet*{BenoistQuint2016} and \citet*{FanEtAl2002}.

\begin{remark}
	For the use of \verb*|natbib| and format of arXiv preprint, it is recommended to use the \verb*|.bst| files \verb*|*nat.bst| or \verb*|*natDOI.bst| at
	
	\href{https://github.com/zfengg/toolkit/tree/master/tex/bst}{https://github.com/zfengg/toolkit/tree/master/tex/bst}.
	
	Otherwise, all the other default \verb*|bst| styles suffices.	
\end{remark}

% ---------------------------------------------------------------------------- %
%                                   reference                                  %
% ---------------------------------------------------------------------------- %
% Get my*.bst from https://github.com/zfengg/toolkit/tree/master/tex/bst
\bibliographystyle{myabbrvnat}
\bibliography{myarticle}\label{sec:ref}

% otherwise, copy .bbl file to here
%\begin{thebibliography}{99999}
%	
%	\bibitem{Falconer2003}K. J. Falconer.
%	{\em Fractal geometry. Mathematical foundations and applications.}  Wiley, 2003.
%		
%\end{thebibliography}


\end{document}
